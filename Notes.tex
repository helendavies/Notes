\documentclass[11pt, oneside]{article}   	% use "amsart" instead of "article" for AMSLaTeX format
\usepackage{geometry}                		% See geometry.pdf to learn the layout options. There are lots.
\geometry{letterpaper}                   		% ... or a4paper or a5paper or ... 
%\geometry{landscape}                		% Activate for rotated page geometry
\usepackage[parfill]{parskip}    		% Activate to begin paragraphs with an empty line rather than an indent
\usepackage{graphicx}				% Use pdf, png, jpg, or eps§ with pdflatex; use eps in DVI mode
								% TeX will automatically convert eps --> pdf in pdflatex		
\usepackage{amssymb}
\usepackage{amsmath}
\usepackage{hyperref}
%SetFonts

%SetFonts


\title{Random stuff}
\author{Me}
%\date{}							% Activate to display a given date or no date

\begin{document}
\maketitle
%\section{}
%\subsection{}

\subsection*{Collision cross section}
This is the area around an atom/molecule that another atom/molecule's centre would have to be inside for a reaction to occur (collision occurs when the distance between the centres of the reactants is less than the sum of their radii). The collision cross section is for one molecule but is dependent on the other molecule too.
The collision cross section between molecule A and B $\sigma_{AB}$ is 
\begin{equation}
\sigma_{AB} = \pi(r_A + r_B)^2
\end{equation}

\url{http://chem.libretexts.org/Core/Physical_and_Theoretical_Chemistry/Kinetics/Modeling_Reaction_Kinetics/Collision_Theory/Collisional_Cross_Section}

\cite{Weltmann2009}


\bibliographystyle{unsrt}
\bibliography{/Users/hld523/Bibliography/MyPapers}

\end{document}  